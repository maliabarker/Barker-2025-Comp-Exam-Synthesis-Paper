%!TEX TS-program = pdflatexmk
%this will be based on the amsart class, which probably causes more pain than necessary
%but gets a lot of the formatting the way I like it
%look for !!!EDIT comments
\documentclass[oneside,12pt]{amsart}
\usepackage{amsaddr}
\usepackage[T1]{fontenc}
\usepackage{amsmath,amsfonts,amsthm,amssymb} %provides all math symbols etc.

%graphics
\usepackage{graphicx}
\usepackage{wrapfig}
\usepackage[margin=0.1cm,textfont=it]{caption}

%set 1" margins:
\usepackage[left=1in,top=1in,right=1in,bottom=1in,footskip = 0.333in]{geometry}

%provide additional table capability
\usepackage{tabularx}
\newcolumntype{L}[1]{>{\raggedright\let\newline\\\arraybackslash\hspace{0pt}}m{#1}} %define left justified column type

%pdf handling (only needed if including PDFs)
%\usepackage{pdflscape} 
%\usepackage{rotating}
\usepackage{pdfpages}

%provides nice verbatim environment and allows verbatim in footnotes
%uncomment as needed
%\usepackage{fancyvrb} 

%control enumerate/itemize spacing
\usepackage{enumitem} 

%for editing phase only:
%\usepackage{todonotes}
%\usepackage{draftwatermark}
%\SetWatermarkText{DRAFT}
%\SetWatermarkScale{5}

%no reason spacing should be anything other than single
%\renewcommand{\baselinestretch}{1}\normalsize %single spaced

%set up header (this has to come before the section patching)
\usepackage{fancyhdr}
\setlength{\headheight}{0.2in}
\pagestyle{fancy}
\fancyhf{}
\lhead{\small \sc Program} %!!!EDIT
\chead{\small \sc \rightmark}
\rhead{\footnotesize \thepage}
\renewcommand{\headrulewidth}{0pt}

% need to redefine section to allow rightmark to work with amsart
\let\origsection\section
\renewcommand{\section}[1]{\sectionmark{#1}\origsection{#1}}
% redefine sectionmark
\renewcommand\sectionmark[1]{\markboth{#1}{#1}}

%fix toc
\setcounter{tocdepth}{3}% to get subsubsections in toc
\makeatletter
\renewcommand{\@pnumwidth}{2em}% default is 1.55em -accommodates wider page numbers
\def\@tocline#1#2#3#4#5#6#7{\relax
  \ifnum #1>\c@tocdepth % then omit
  \else
    \par \addpenalty\@secpenalty\addvspace{#2}%
    \begingroup \hyphenpenalty\@M
    \@ifempty{#4}{%
      \@tempdima\csname r@tocindent\number#1\endcsname\relax
    }{%
      \@tempdima#4\relax
    }%
    \parindent\z@ \leftskip#3\relax \advance\leftskip\@tempdima\relax
    \rightskip\@pnumwidth plus4em \parfillskip-\@pnumwidth
    #5\leavevmode\hskip-\@tempdima
      \ifcase #1
       \or\or \hskip 1em \or \hskip 2em \else \hskip 3em \fi%
      #6\nobreak\relax
    \dotfill\hbox to\@pnumwidth{\@tocpagenum{#7}}\par
    \nobreak
    \endgroup
  \fi}
\makeatother

%section numbering
\renewcommand{\thesubsection}{\alph{subsection}}
\renewcommand{\thesubsubsection}{\roman{subsubsection}}
\makeatletter
\renewcommand{\p@subsection}{\thesection.}
\renewcommand{\p@subsubsection}{\thesection.\thesubsection.}
\makeatother

%Define invisible section and subsection for including PDF docs
\newcommand\invisiblesection[1]{%
  \refstepcounter{section}%
  \refstepcounter{page}  %%only if numbering pages within sections
  \addcontentsline{toc}{section}{\protect{\thesection.}{\hskip 1em}#1}%
  \sectionmark{#1}}
  
\newcommand\invisiblesubsection[1]{%
  \refstepcounter{subsection}%
  \addcontentsline{toc}{subsection}{\thesubsection.{\hskip 1em}#1}%
  }
  
%page numbering by section
\numberwithin{page}{section}
\renewcommand{\thepage}{\thesection-\arabic{page}}

%% Make sure that page starts from 1 with every \section
\usepackage{etoolbox}
\makeatletter
\patchcmd{\@sect}% <cmd>
  {\protected@edef}% <search>
  {\def\arg{#1}\def\arg@{section}%
   \ifx\arg\arg@\stepcounter{page}\fi%
   \protected@edef}% <replace>
  {}{}% <success><failure>
\makeatother

%hyperref load has to come after section  patching
\usepackage[breaklinks=true,colorlinks=true,citecolor=red]{hyperref}
\newcommand{\nlhref}[1]{\href{#1}{\nolinkurl{#1}}} %automatically create exact url with href

%create boolean for toggling inclusion of backmatter
\providetoggle{backmatter}
\settoggle{backmatter}{false} %set to true to toggle on/false off %!!!EDIT

%bibliography handling
\usepackage[citestyle=authoryear-comp,bibstyle=numeric, natbib=true, backend=bibtex,hyperref=true,maxbibnames=99,maxcitenames=2]{biblatex}
\addbibresource{Main.bib} %!!!EDIT

% private defs
\def\mf{\mathbf}
\def\mb{\mathbb}
\def\mc{\mathcal}
\newcommand{\bbar}[1]{\mf{\bar{#1}}}
\newcommand{\bhat}[1]{\mf{\hat{#1}}}
\newcommand{\refeq}[1]{Equation  (\ref{#1})}
\newcommand{\refsec}[1]{\S\ref{#1}} 
\newcommand{\reftable}[1]{Table \ref{#1}} 
\newcommand{\refch}[1]{Chapter  \ref{#1}} 
\newcommand{\reffig}[1]{Figure \ref{#1}}
\newcommand{\refcode}[1]{Listing \ref{#1}}
\newcommand{\intd}[1]{\ensuremath{\,\mathrm{d}#1}}
\newcommand{\leftexp}[2]{{\vphantom{#2}}^{#1}\!{#2}}
\newcommand{\leftsub}[2]{{\vphantom{#2}}_{#1}\!{#2}}
\newcommand{\fddt}[1]{\ensuremath{\leftexp{\mathcal{#1}}{\frac{\mathrm{d}}{\mathrm{d}t}}}}
\newcommand{\fdddt}[1]{\ensuremath{\leftexp{\mathcal{#1}}{\frac{\mathrm{d}^2}{\mathrm{d}t^2}}}}
\newcommand{\omegarot}[2]{\ensuremath{\leftexp{\mathcal{#1}}{\boldsymbol{\omega}}^{\mathcal{#2}}}}
\DeclareMathOperator{\rank}{rank}

\title[A Review of Tidal Disruption]{A Review of the Theory and Application of Finding Tidal Disruption in Extrasolar Systems}
\author{Malia Barker}
\address{Department of Computer Science, Boise State University}

\begin{document}

\begin{abstract}
Since the first exoplanet was officially detected in the 1990s, the number of planets beyond our Solar System has skyrocketed. With nearly 6,000 confirmed exoplanets, we can now explore the intricate orbital dynamics of these distant worlds, studying how they interact with their stars and neighboring planets. Among the most intriguing are Ultra-Hot Jupiters (UHJs)—gas giants similar in size and composition to Jupiter, yet with an unexpected twist. Unlike any planet in our Solar System, UHJs orbit their stars at extreme proximity, completing an entire orbit in less than three days. These tight orbits give rise to intense gravitational interactions between the planet and its star, causing the planet to spiral gradually into its star until it reaches the point of no return and is disrupted. 

UHJs are particularly interesting to study because their extreme orbits make them easier to observe. Due to their large size and close proximity to their stars, many of these planets regularly pass in front of their host stars, casting an observable shadow in an event known as a transit. By precisely measuring the time between transits, we can track small changes in the orbits that may arise due to a decreasing orbital period.

The search for such doomed worlds is a collaborative effort across multiple scientific disciplines. Theorists investigate the stellar physics and orbital dynamics driving planetary in-spiral, while observers track shrinking orbital periods using both ground- and space-based telescopes—including some on Boise State’s campus. These changes in orbital decay are subtle, on the order of milliseconds per Earth year, making unguided searches inefficient. My work bridges these efforts by developing Susie, a software package that streamlines the detection of these small orbital changes in observational data, helping astronomers more efficiently identify and study planets undergoing tidal decay.
\end{abstract}

\maketitle

\clearpage

\section{Table of Contents}
\renewcommand\contentsname{}
\tableofcontents

\clearpage

\section{Synthesis Article}

% \subsection{Introduction}\label{sec:intro}

% \subsubsection{Orbital Decay due to Tidal Interactions}\label{sec:orbdecay}
% - Intro to Exoplanets, transits, tidal interactions and orbital decay due to tidal forces (Tidal Decay)

% \subsubsection{Tidal Disruption}\label{sec:disruption}
% - Intro to Tidal disruption, Roche limit, disruption and accretion, we don't know which happens (CAN TIE MULTIPLE PAPERS INTO THIS)


% \subsection{Theory}\label{sec:theory}

% \subsubsection{Tidal Dissipation}\label{sec:tidaldiss}

% \subsubsection{Constrainting Q^{*}}\label{sec:Q*}
% - Tidal Dissipation/Constraining Q*' & Engulfment rate
%     - For constraining Q*', can use theoretical stellar models and an interpolable grid OR observations of tidal decay, which we can then use the decay rate to calculate an estimated value of Q*' (eq in Adams 2024)
%     - Important to constrain Q* because this can tell us how many planets are falling into their stars and therefore how often we should be looking. For example, if only 1 planet is engulfed by star a year in the milky way, then our chances of finding this are pretty low. However, if this is larger, then the fact we haven't found it is weird, right?
%     - However, there is also the possibility that we cannot constrain Q* to any one number, especially if the value depends on stellar properties (such as brought up in Ogilvie paper). Then what does that mean for using Q* to predict planetary engulfment rate? Would we still have a general range of number of planets engulfed in a year or does this idea just go out the window?


% \subsection{Observation}\label{sec:obs}

% \subsubsection{Timing Observations}\label{sec:timingobs}
% - How it is taken (different telescopes, different uncertainties)
% - For ground based telescopes, can have composite data (from Elisabeth's ppr)
%     - Amateur data such as Exoplanet Transit Database
    
% \subsubsection{Optimizing Timing Observations}\label{sec:optimizingobs}
% - Can find best way to observe using analytical delta BIC from Jackson 2023 (uncertainties based on ground vs space scopes, frequency and total number of obs)
% - smaller tidal dissipation parameter means larger tidal decay rate, and smaller Q* values can maybe be seen in later-type stars with deeper convective zones (CAN TIE IN JACKSON AND OGILVIE PAPER)
% - MS stars tend to be smaller and therefore have deeper transit depths, leading to better light curves and less error in the mid-time, but if cooler are also usually dimmer which increase uncertainty
% - In adams, she mentions that UHJs are usually found around young MS stars. However, the only confirmed tidally decaying UHJ (WASP-12 b) is possibly around red giant and maybe even Kepler-1658 b is around an evolved star, so maybe not?

% \subsubsection{Timing Observation Processing}
% - Fitting light curves (lightkurve, pylight-curve, emcee) and comparing how these differ, how one may perform better over another, how often they are used

% \subsubsection{Timing Data Management}
% - identifying duplicate times (pg 9 of Adams 2024 has a list of when repeats are acceptable or when they are duplicates)
% - identifying composite times
% - identifying false conversion from JD/UTC to BJD/TDB
% - obviously a lot of different issues in storing and retrieving timing data, which is an issue for this type of research that needs very precise timing data. A way to remedy a lot of these issues is brought up in conclusions of Adams 2020 paper

% \clearpage 

\subsection{Introduction}\label{sec:intro}

\subsubsection{Background on Ultra-Hot Jupiters (UHJs)}
% INTRODUCTION OUTLINE
% - DONE discovery of exoplanets (what is an exoplanet)
% - DONE discovery of hot jupiters (what is a hot jupiter)
% - DONE formation theories of hot jupiters
%     - DONE in situ (grew in their locations)
%     - DONE ex situ (eccentricity excitation and tidal migration), (pull in TIC 393818343 b as an example of testing out these formation theories)
% - DONE orbital evolution of hot jupiters (we should expect to see them falling into their stars, especially because UHJs tend to exist around younger stars, so what is happening to them?) 
% - DONE observations and data collection of hot jupiters (transits & transit light curves)

% - looking for key characteristics of orbital evolution with transit light curves to find out
%     - how did hot jupiters get to the current orbits
%     - how do hot jupiters interact with their stars over time/how do their orbits change
%     - do hot jupiters eventually destruct/why are there such few around older stars/what is happening to them???

In 1992, Aleksander Wolszczan and Dale Frail published the first confirmed detection of two planets orbiting the highly active neutron star PSR 1257+12 \citep{wolszczan1992planetary}. While the existence of planets beyond our solar system had been heavily debated, this discovery provided definitive proof and sparked widespread curiosity among astronomers, launching the modern search for exoplanets. Since then, planetary science has grown into a major subfield of astrophysics. Nearly 6,000 exoplanets—spanning a wide range of masses, radii, orbital periods, and atmospheric conditions—have been discovered and confirmed, with many more yet to be found. Today, almost every observatory contributes to exoplanet research, from high-profile missions like NASA's James Webb Space Telescope to amateur astronomers who add their nightly observations to public archives. With so many extrasolar systems vastly different from our own, there is still much to explore and understand.

Among the many known exoplanets, hot Jupiters are of particular interest. Just a few years after the first exoplanet discovery, \citet{mayor1995pegasi51b} confirmed the first Jupiter-mass planetary companion, 51 Pegasi b. These exoplanets—classified as hot Jupiters—are similar in size and composition to Jupiter but stand out due to their extremely close orbits around their host stars. For example, 51 Pegasi b completes one full revolution around its star in approximately four days, placing it more than seven times closer to its host star than Mercury is to the Sun in our own solar system. Most hot Jupiters complete one orbit around their host star in ten days or less, a duration known as the orbital period \citep{fortney2021hotjupiters}. However, current planetary formation theories do not support the idea that a Jupiter-mass planet could form so close to its star. This contradiction has led to an ongoing debate: how did hot Jupiters end up in such tight orbits?

\begin{figure}[htbp]
    \centering
    \includegraphics[width=0.9\linewidth]{figs/hj_formation_theories.jpg}
    \caption{Figure 1 from \citet{fortney2021hotjupiters} illustrates the leading formation theories of close-in gas giants: in-situ formation, disk migration, and high-eccentricity tidal migration.}
    \label{fig:formation-theories}
\end{figure}

In the early stages of a star's life, gas and small particles orbit around it, forming a protoplanetary disk—the birthplace of planets. \citet{fortney2021hotjupiters} and \citet{dawson2018origins} propose three main mechanisms for giant planet formation within this disk: in-situ formation, where the planet forms at its present-day orbital period; disk migration, where interactions with surrounding material drive the planet’s orbit inward; and high-eccentricity migration, where gravitational interactions with another body excite the planet’s orbit, followed by recircularization, which results in its current orbital period.

In situ formation, where a planet forms at its current orbital period close to its host star, can occur through core accretion or gravitational instability. 

Core accretion occurs when a large planetary core—about 10 Earth masses—accumulates a gaseous envelope around it to form a gas giant \citep{helled2013giant}. \citet{dawson2018origins} argue that while core accretion can amass enough gas to form a gas giant, the required initial 10 Earth mass core is difficult to form within the short orbital periods of hot Jupiters and the limited lifespan of the protoplanetary disk (which supplies the gas for accretion). Similarly, \citet{fortney2021hotjupiters} highlights several challenges in forming a large planetary core, including insufficient material in the protoplanetary disk to build the core, the difficulty of smaller cores merging into a larger one due to the disk's dynamics, and stalled pebble accretion by an existing core before it reaches the required mass of 10 Earth masses. These challenges align with the theories presented by \citet{dawson2018origins}, supporting the overall conclusion that core accretion may not be a viable mechanism for hot Jupiter formation.

Gravitational instability occurs when self-gravitating clumps of matter form directly within the protoplanetary disk \citep{helled2013giant}. Theories of gravitational instability face similar issues to the core accretion theory. \citet{dawson2018origins} note that the high temperatures and short orbital periods of hot Jupiter regions prevent gas from collapsing under its own gravity, making this mechanism unlikely as well. As a result, neither theory of close-in formation is widely accepted.

Outside the orbital period range of hot Jupiters, both core accretion and gravitational instability become much more viable mechanisms for planet formation. This gives rise to another theory suggesting that hot Jupiters may initially form farther from their current orbits and migrate inward due to torques exerted by the protoplanetary disk, known as disk migration. However, \citet{fortney2021hotjupiters} and \citet{dawson2018origins} emphasize that migration is highly sensitive to disk conditions. If other forces fail to halt migration, the planet may continue inward past the hot Jupiter zone, ultimately being engulfed by its host star.

Another proposed mechanism for inward migration within the protoplanetary disk is high-eccentricity tidal migration. This process occurs when another object in the disk—most likely another forming planet—perturbs the gas giant’s orbit. As the planet settles into an elongated orbit, gravitational interactions with its host star eventually cause the orbit to recircularize, with the final semimajor axis—the furthest distance between the planet and its star in its orbit—given by

\begin{equation}
    a_{\text{final}} = a(1 - e^2)
    \label{eq:final_semimajor_axis}
\end{equation}

where $a_{\text{final}}$ is the planet’s final semimajor axis after recircularization, $a$ is the original semimajor axis of its elongated orbit, and $e$ is the orbital eccentricity \citep{dawson2018origins}.

This theory is strongly supported by observations of warm Jupiters—planets thought to be precursors to hot Jupiters, with orbital periods between 10 and 200 days. Many warm Jupiters have eccentric orbits, and when recircularized, their final semimajor axes correspond to orbital periods of less than 10 days, aligning with the expected outcome of high-eccentricity tidal migration.

Consider the case of a recently studied exoplanet, TIC 393818343 b. In a recent paper, \citet{sgro2024confirmation} used ground-based observations to confirm that the planet has a non-zero eccentricity and an orbital period of approximately 16 days. If its orbit were to eventually recircularize, applying the observed orbital parameters to equation \ref{eq:final_semimajor_axis} yields a final orbital period of about 8 days—placing it well within the hot Jupiter region.

Across all formation theories, it is possible that planets initially form beyond their current orbital periods, migrate inward through disk migration or high-eccentricity tidal migration, and then continue migrating even after the protoplanetary disk dissipates due to gravitational interactions with their host stars. This continued inward migration may explain the existence of ultra-hot Jupiters (UHJs)—a subset of gas giants with orbital periods of less than three days.

UHJs provide a unique window into the final stages of hot Jupiter orbital evolution. Due to their large masses and extreme proximity to their host stars, UHJs exert strong gravitational forces, inducing significant tidal bulges on their stars \citep{jackson2023metrics}. A tidal bulge is a deformation of the star's shape, where the gravitational pull from the planet causes the star to stretch slightly, similar to how tides are raised on the Earth by the Moon. The tidal bulge and the exoplanet tend to move toward synchronization; however, if the planet orbits faster than the tidal bulge rotates—causing the bulge to lag behind—angular momentum is transferred from the planet’s orbit to the star \citep{ogilvie2014tidal}. This loss of angular momentum causes the planet's orbit to shrink, and the planet gradually spirals into its host star, a process known as orbital decay due to tidal interactions, or tidal decay. If no other mechanism intervenes to stall this process, these planets are expected to eventually undergo mass loss and ultimately be destroyed by their stars. A more in-depth review of this process is presented in Section \ref{sec:tidaldissipation}.

% \textbf{TODO:} Should I add in information about merger outcomes using this paper: \citet{metzger2012optical}. This would cover stable/unstable mass loss, roche lobe overflow
% - fills roche lobe below the stellar surface -> direct impact merger event
%     - this has been potentially observed by \citep{de2023infrared}
% - fills roche lobe above the stellar surface -> 
%     - disrupted into an accretion disk around star called a tidal disruption merger event
%     - gradually transfers mass over a long time scale called stable mass transfer

In line with theories of tidal decay and planetary destruction, observations suggest that UHJs do not remain in stable orbits indefinitely. Statistical studies indicate that these planets are more commonly found around younger stars than the broader population of planet-hosting stars \citep{hamer2019hot}, hinting that many UHJs may eventually be lost to tidal decay. However, only one confirmed hot Jupiter, WASP-12 b, has been observed undergoing tidal decay \citep{yee2019orbit}. This raises the possibility that some UHJs are actively spiraling into their host stars, but the timescales and underlying mechanisms driving or delaying this process remain uncertain. Investigating the orbital evolution of UHJs is therefore essential to understanding their long-term stability and ultimate fate.

\begin{figure}[htbp]
    \centering
    \includegraphics[width=0.6\linewidth]{figs/winn_fig1.png}
    \caption{Figure 1 from \citet{winn2010transits} illustrates an exoplanet transit and occultation edge-on. During a transit, the planet passes in front of its host star, blocking starlight and producing a characteristic dip in flux over time. During an occultation, the planet moves behind the star, obscuring its reflected light and producing a smaller flux dip.}
    \label{fig:winnfig1}
\end{figure}

An effective method for studying orbital evolution in UHJs is transit photometry. When an exoplanet passes in front of its host star, it temporarily blocks a fraction of the star’s light (also known as flux), producing a characteristic dip in brightness—an event known as a transit. Similarly, when the planet moves behind the star, the light it reflects is briefly obscured, causing a smaller dip—an event known as an occultation (see Figure \ref{fig:winnfig1} for a visualization).

UHJs are particularly well-suited for transit observations due to their large radii, which produce deep, easily detectable transit signals, and their short orbital periods, which allow for frequent monitoring. Because these planets complete orbits in just a few days, multiple transits can be observed over relatively short timescales, providing ample data to track orbital changes.

By precisely measuring transit times, astronomers can determine an exoplanet’s orbital period. This is done by recording the midtime—the moment the planet reaches the center of its transit across the star—and calculating the interval between consecutive midtimes. Even slight variations in these timings can reveal the subtle effects of tidal interactions, offering key evidence for ongoing orbital decay. A more detailed discussion of the transit method and its role in detecting orbital evolution is presented in Section \ref{sec:obs-photo}.

Understanding the orbital evolution of UHJs is critical for constraining the long-term fate of close-in giant planets and testing theoretical models of planetary migration and tidal interactions. While only one exoplanet, WASP-12 b, has been definitively observed undergoing orbital decay, continued transit observations of other UHJs may reveal additional cases, shedding light on the mechanisms that govern their ultimate destruction. By leveraging high-precision photometric observations and statistical modeling, we can refine our understanding of planetary evolution. The following sections will explore the theoretical framework governing planetary orbital evolution and the observational techniques used to detect tidal decay.


% ————————————————————————————————————————————————————————————————
% —————————————————————    NEW SECTION    —————————————————————
% ————————————————————————————————————————————————————————————————


\subsection{Mechanisms of Tidal Dissipation in Stars and Giant Planets}\label{sec:tidaldissipation}

Tidal dissipation is the transfer of energy and angular momentum from one body to another due to tidal forces. Where tidal dissipation occurs plays a role in shaping orbital evolution of an extrasolar system. Dissipation within the planet primarily drives tidal circularization, smoothing out orbital eccentricities, such as in the process of a warm Jupiter becoming a hot Jupiter. Meanwhile, dissipation within the star facilitates angular momentum transfer from the planetary orbit to the stellar spin, leading to tidal decay. Understanding these mechanisms is key to predicting the long-term fate of UHJs and their survival timescales.

\begin{figure}[htbp]
    \centering
    \includegraphics[width=0.7\linewidth]{figs/ogilvie_fig1.png}
    \caption{Figure 1 from \citet{ogilvie2014tidal} illustrates the geometry of tidal interactions between two bodies. The spherical polar coordinates ($\theta$, r, and $\phi$, in green) represent the orbital planes, centered on body 1 with mean radius $R_1$ and mass $M_1$, and body 2 with mass $M_2$, separated by a distance $d$. The spin angular velocity vector, $\Omega_s$, and the orbital angular velocity vector, $\Omega_o$, are shown in red. The solid black line represents the ellipsoidal bulge raised on body 1 by body 2, while the dotted black line illustrates the unaltered, spherical shape of body 1.}
    \label{fig:ogilvie-fig1}
\end{figure}

The tidal potential, which describes the gravitational influence one body exerts on another, is typically expanded into multipole moments using spherical harmonics. The simplest case, the monopole moment ($l = 0$), treats the body as a point mass, assuming all its mass is concentrated at a single point. This approximation is useful for the gravitational interaction at a large scale but does not account for the internal structure of the body. A more detailed description of bodies with irregular shapes, such as those affected by tidal forces, requires the quadrupole moment ($l = 2$). The quadrupole moment accounts for the body's deviation from spherical symmetry, as seen in stars or ultra-hot Jupiters (UHJs) that develop tidal bulges due to their proximity to eachother.

Figure \ref{fig:ogilvie-fig1} illustrates the interaction between two bodies, where body 1 is treated as a quadropole with a tidal bulge raised on it by body 2, which is treated as a monopole. In Figure \ref{fig:ogilvie-fig1}, the tidal bulge and body 2 are aligned in the orbital plane. However, in reality, the tidal bulge raised on body 1 is misaligned with the axis connecting the two bodies, leading to tidal forces that transfer angular momentum and dissipate energy \citep{ogilvie2014tidal}.

In a UHJ system, both the planet and the star raise tidal bulges on each other. Over time, the misalignment of the UHJ's tidal bulge and the star's rotation can drive the planet toward tidal locking, where the rotation period of the planet synchronizes with its orbital period, causing one hemisphere to always face its host star. For example, HAT-P-7 \citep{helling2019understanding} exhibits significant temperature differences on its day and night sides, indicative of tidal locking. The misalignment of the host star's tidal bulge and the planet's orbital period leads to a transfer of angular momentum from the planet's orbital angular momentum to the star's spin angular momentum, resulting in tidal decay.

\begin{table}[htbp]
    \centering
    \includegraphics[width=0.9\linewidth]{figs/ogilvie_tbl1.png}
    \caption{Table 1 from \citep{ogilvie2014tidal} shows the quadrupolar components of the tidal potential.}
    \label{tab:ogilvie-tbl1}
\end{table}

The tidal potential can be broken down into several components, as detailed in Table \ref{tab:ogilvie-tbl1}. These include different contributions to the tidal interactions, characterized by the spherical harmonic order $l$ and the longitudinal structure of the bulge, represented by the index $m$. For instance, when $m = 0$, the tidal bulge is axisymmetric, while non-axisymmetric cases occur when $m = 1$ or $m = 2$. The time-dependence of the tidal deformation is captured by the index $n$, with $n = 0$ representing a static tide, and $n > 0$ indicating periodic oscillations due to orbital motion. The tidal amplitude, $|A|$, quantifies the strength of each tidal component, with various types of tides—such as static, asynchronous, eccentricity, and obliquity tides—arising under different conditions, further affecting the system’s evolution.

A static tide, corresponding to $n=0$ and $m=0$, represents a permanent, axisymmetric tidal bulge that remains fixed relative to the body, producing a constant deformation. In contrast, an asynchronous tide, with $l=m=n=2$, occurs when the body’s spin is not synchronized with its orbit, causing the tidal bulge to become misaligned with the orbital motion.

An eccentricity tide arises when the body follows an elliptical orbit. The amplitude $|A|$ of this tide is proportional to the orbital eccentricity $e$, causing periodic variations in the tidal forces as the body moves through its orbit at varying speeds. An obliquity tide is driven by a misalignment between the body’s spin axis and its orbital plane. The amplitude $|A|$ of this tide is proportional to the obliquity $i$, the angle of the body's tilt relative to its orbital plane, which introduces additional tidal components that vary with the tilt of the body.

These different tidal components are critical for understanding the intricate evolution of planetary systems, particularly in hot Jupiter systems where tidal interactions play a significant role in the dynamics.

\begin{figure}[htbp]
    \centering
    \includegraphics[width=0.5\linewidth]{figs/ogilvie_fig2.png}
    \caption{Figure 2 from \citet{ogilvie2014tidal} depicts the tidal forcing frequencies for the quadrupolar tidal components listed in Table \ref{tab:ogilvie-tbl1}. The vertical axis represents the tidal frequency in the fluid frame, which is the angular frequency ($\hat{\omega}$) measured in a reference frame rotating with the body, normalized by the body’s spin frequency ($\Omega_s$). This ratio indicates how the tidal forces compare to the body’s own rotation. The horizontal axis shows the orbital frequency ($\Omega_o$), which is the rate at which body 1 orbits body 2, normalized by the spin frequency ($\Omega_s$). This axis illustrates the relative speed of the body’s orbit compared to its spin. The dashed vertical line at $\frac{\Omega_o}{\Omega_s} = 1$ represents the alignment of the orbital and spin frequencies (e.g., tidal locking). Each colored line, labeled by a set of quadrupolar components from Table \ref{tab:ogilvie-tbl1}, has a slope of $n$ and a y-intercept of $-m$. The dotted lines indicate the frequency range at which inertial waves can be excited within the rotating body.}
    \label{fig:ogilvie-fig2}
\end{figure}

Figure \ref{fig:ogilvie-fig2} visualizes the seven sets of quadrupolar tidal components listed in Table \ref{tab:ogilvie-tbl1}, providing a comprehensive view of the different types of tidal interactions.

The dashed vertical line in Figure \ref{fig:ogilvie-fig2} represents orbit and spin angular momenta alignment. As an example, let's assume an asynchronous tide in a hot Jupiter, which plays a role in energy dissipation early in a hot Jupiter's evolution, indicated by the yellow line labeled by $2, 2, 2$. When the tide reaches the synchronous state at $\frac{\Omega_o}{\Omega_s} = 1$, then $\frac{\hat{\omega}}{\Omega_s} = 0$, meaning the tidal bluge stops osciallating within the planet. This describes the mechanism of tidal locking, when the tidal bulge of the planet permanently faces the star once the spin and orbital frequencies align. Once in a tidally locked state, this mode no longer plays a role in energy dissipation in the planet (although other modes, especially those raised in the star, will). As previously mentioned, most hot Jupiters follow this track and once they enter the UHJ region, are assumed to be tidally locked.

Different tides may dominate at different times in a hot Jupiter's evolution. If the planet starts with a high orbital eccentricity, then eccentricity tides may dominate early on in an attempt to circularize the orbit, such as in a high-eccentricity migration scenario. If the planet is initially spinning very fast, then the asynchronous tide may act to slow the rotation and align with the orbit (as mentioned above). If the planet is tilted on its rotational axis, then obliquity tides may act to realign the spin with its orbital plane. As a hot Jupiter moves through its evolution towards a UHJ with an aligned, tidally locked, and circular orbit, the static tide dominates, as seen by the red line in Figure \ref{fig:ogilvie-fig2}.

The dotted horizontal lines in Figure \ref{fig:ogilvie-fig2} represent the range of frequencies at which inertial waves can be excited inside a rotating body, defined by the condition $|\frac{\hat{\omega}}{\Omega_s}| < 2$. Notably, all seven sets of tidal components in the figure fall within this range, indicating that inertial wave excitation is relevant for all these modes.

Inertial waves arise due to the Coriolis force, which acts on fluids moving within a rotating system, causing them to be deflected (fun fact: this is why hurricanes spin in opposite directions in different hemispheres on Earth). In a fully convective body, such as a star, these waves can freely propagate. When a tidal component’s frequency ($\frac{\hat{\omega}}{\Omega_s}$, the y-axis) falls within the inertial wave regime (marked by the dotted horizontal lines in Figure \ref{fig:ogilvie-fig2}), it can resonate with fluid motions inside the planet, significantly enhancing tidal dissipation \citep{ogilvie2014tidal}. This mechanism may explain the high tidal dissipation rates observed in UHJ systems, which play a crucial role in their orbital evolution and eventual fate.

Inertial waves may also propagate in planets differently depending on their internal structure. \citet{ogilvie2014tidal} discusses how larger cores and greater density contrasts between layers lead to stronger inertial wave-driven tidal dissipation. This has implications for different UHJ formation scenarios. If UHJs form in-situ, they are expected to have massive cores ($\geq$ 10 Earth masses) \citep{fortney2021hotjupiters}, \citep{dawson2018origins}, leading to enhanced tidal dissipation, and may explain how these planets moved into their orbital ranges so quickly.

% Explaining angular momentum
Over time, tidal forces drive a system toward tidal equilibrium, in which both bodies achieve aligned, synchronous rotation and a circular orbit, and there is no further transfer of angular momentum. However, in systems with large mass differences, such as UHJ systems, this equilibrium may not be possible because the total angular momentum of the system is insufficient.

The total angular momentum of a system is the sum of orbital angular momentum ($L_o$)–the angular momentum associated with the planet's orbit around its host star—and the spin angular momentum ($L_s$)–the angular momentum from the planet’s rotation about its own axis.
For tidal equilibrium to be reached, the total angular momentum ($L = L_o + L_s$) must exceed a critical threshold ($L_c$). If $L < L_c$, the system continues evolving instead of settling into a stable state.

In extreme mass ratio systems, the smaller body contributes very little spin angular momentum because its moment of inertia is small. As a result, the total angular momentum is much lower than in systems with two similar-mass bodies (such as binary stars), where both contribute significant spin angular momentum. If the system cannot reach equilibrium, tidal interactions continue to transfer angular momentum between the planet’s orbit and the star’s spin, causing the planet’s orbit to decay.

% In circular orbits, rotation speed matches orbit and the spin axis aligns to straight-up direction relative to the orbital plane. However, in eccentric orbits, the tidal forces are strongest at the pericenter, so the spin usually synchs at the fastest part of the orbit, entering a state of psuedo-synchronization. If we consider the warm to hot jupiter theory represented in \citep{fortney2021hotjupiters} and \citep{dawson2018origins}, if we start with an eccentric orbit, then the planet will enter pseudo-synchronization 

\begin{figure}[htbp]
    \centering
    \includegraphics[width=0.5\linewidth]{figs/ogilvie_fig10.png}
    \caption{Figure 10 from \citet{ogilvie2014tidal} illustrates the tidal evolution of a planet-star system, assuming a circularized orbit with $e=0$. The vertical axis represents the stellar spin angular velocity, measured in units of the critical angular velocity, $\Omega_c$, which corresponds to the critical angular momentum, $L_c$. The horizontal axis shows the planetary orbital angular velocity, also in units of the critical angular velocity $\Omega_c$. The solid lines are contours of the total angular momentum of the system ($L$), with arrows indicating the direction of evolution, where energy is dissipated. The dashed line marks the point at which the stellar spin and planetary orbit are synchronized. Black points represent moments of tidal equilibrium, where $L \geq L_c$.}
    \label{fig:ogilvie-fig10}
\end{figure}

Figure \ref{fig:ogilvie-fig10} shows different trajectories of a planet tidally interacting with its host star, based on the total angular momentum of the system, expressed in terms of the critical angular momentum ($\frac{L}{L_c}$), assuming the orbit is circularized, as seen in a UHJ system.

If the system's angular momentum $L$ is greater than or equal to the critical angular momentum $L_c$, the system can remain in equilibrium for an extended period. The system attempts to evolve toward tidal equilibrium (black dots) and synchronization (dashed line), as indicated by the directional arrows on the contour lines. Above the synchronization line, the system has more spin angular momentum, and the star spins down as energy is dissipated. Below this line, the system has more orbital angular momentum, and tidal equilibrium is not achieved ($L < L_c$). In this case, angular momentum is transferred from the orbit to the stellar spin, leading to a shrinking orbital period and stellar spin-up—this process is known as tidal decay.

Tidal interactions play a crucial role in shaping the orbital evolution of UHJs, with tidal dissipation being central to this process. Accurately quantifying the tidal dissipation rate, however, has proved to be difficult. One key parameter used to describe tidal dissipation is the tidal quality factor, $Q_*$, a dimensionless, non-literal parameterization of the tidal response that represents how efficiently the system dissipates tidal energy \citep{ogilvie2014tidal}. While $Q_*$ is commonly used in theoretical settings, it is difficult to measure directly. Instead, a more frequently employed metric is the modified tidal quality factor, $Q_*' = \frac{Q_*}{k_2}$ \citep{patel2022constraining}, where $k_2$ is the Love number that quantifies how a body deforms in response to tidal forces. The modified tidal quality factor accounts for this deformation, making it easier to measure from observational data. A larger value of $Q'_*$ indicates weaker tidal dissipation, while a smaller value suggests stronger dissipation.

The value of $Q'_*$ depends on the tidal frequency and the quadrupolar components of the spherical harmonics ($l$ and $m$), meaning that the value of $Q'_*$ can vary with the internal structure of the body \citep{ogilvie2014tidal}. For example, in the case of binary star recircularization, $Q'_*$ is typically found to be around $10^6$ \citep{ogilvie2007tidal}. In contrast, for UHJs, this value can vary significantly, with WASP-12 b having $Q^{'}_{*} = 1.8 \times 10^5$ \citep{yee2019orbit}.

\begin{table}[htbp]
    \centering
    \includegraphics[width=0.65\linewidth]{figs/adams_tbl8.png}
    \caption{Adapted from Table 8 of \citet{adams2024doomed}, the observed change in orbital period over time ($\dot{P}$) and the calculated change in orbital period over time (calc $\dot{P}$) assuming a modified tidal factor like WASP-12 b's ($Q'_* = 1.8 \times 10^5$ \citep{yee2019orbit}) for four hot Jupiter exoplanets.}
    \label{tab:adams-tbl8}
\end{table}

Many statistical models have traditionally assumed a constant value for the modified tidal quality factor (see Section \ref{sec:models} for an in-depth review of these models and assumptions). However, such an assumption can significantly reduce model accuracy, as $Q^{'}_{*}$ depends intricately on stellar age, mass, and internal structure.

The useful approximation for $Q'_*$ was introduced by \citet{goldreich1966q}:

\begin{equation}
    Q'_* = - \left( \frac{27 \pi}{2} \right ) \left( \frac{dP}{dt} \right )^{-1} \left( \frac{M_p}{M_s} \right ) \left( \frac{a}{R_*} \right )^{-5}
\end{equation}\label{eq:q*}

This equation highlights the inverse relationship between $Q'_*$ and the rate of orbital period change, $\frac{dP}{dt}$ or $\dot{P}$. Using this approximation, \citet{adams2024doomed} calculated the orbital period decay rate for a collection of hot Jupiters under the assumption that their tidal dissipation rate matches that of WASP-12 b. Table \ref{tab:adams-tbl8} compares these calculated values with observed $\dot{P}$ values for four of these systems. The discrepancies between the systems emphasize that a single, constant $Q'_*$ value is not sufficient to accurately model tidal dissipation across different exoplanetary systems.

\begin{figure}[htbp]
    \centering
    \includegraphics[width=0.8\linewidth]{figs/adams_fig11.jpg}
    \caption{Figure 11 from \citet{adams2024doomed} presents both observational and theoretical values of the modified tidal quality factor, $Q^{'}_*$, for various systems. Red triangles represent the $3\sigma$ lower limits of $Q^{'}_*$, calculated using equation \ref{eq:q*}. Systems with negative orbital periods ($\dot{P}$), indicating orbital decay, but with insignificant detections (i.e., $<3\sigma$), are marked with open blue circles. In contrast, the two systems exhibiting significant negative values of $\dot{P}$, WASP-12 b and TrES-1 b, are represented by solid blue circles. The theoretical estimates of $Q^{'}_*$, derived from the grids provided by \citet{weinberg2023orbital}, are shown as purple plus signs. For more details on the system parameter values, refer to \citet{adams2024doomed}.}
    \label{fig:adams-fig11}
\end{figure}

One major challenge in studying tidal dissipation is that measuring $Q'_*$ directly requires additional observational assumptions. To mitigate this, an alternative approach is to model the internal structure and interactions within host stars to predict $Q'_*$ theoretically. \citet{weinberg2023orbital} developed non-linear models to construct an interpolable grid of $Q'_*$ values based on stellar parameters such as mass and age. Using this grid, \citet{adams2024doomed} estimated theoretical $Q'_*$ values for several exoplanetary systems and compared them to observationally derived values. These results are presented in Figure \ref{fig:adams-fig11}.

While the observed values of $Q'_*$ (blue circles) and the theoretical predictions (purple plus signs) appear to differ significantly, it is important to consider two key factors. First, the grid used by \citet{adams2024doomed} was relatively coarse, meaning finer parameter steps could yield more accurate approximations. Second, stellar parameters, particularly stellar age, are notoriously difficult to constrain. For instance, during long evolutionary phases such as the main sequence or giant phase—lasting millions to billions of years—observable changes in a star are minimal, making age estimates highly uncertain.

Beyond informing orbital decay rates, knowledge of $Q'_*$ has broader implications. In particular, it plays a crucial role in estimating the galactic engulfment rate—the number of planets engulfed by their host stars per year across the Milky Way. Since this rate scales with the tidal dissipation rate \citep{jackson2023metrics}, constraining $Q'_*$ allows for a rough estimate of how frequently such merger events occur. However, as research progresses, it is becoming increasingly clear that $Q'_*$ is not a universal constant, meaning the assumption of a single, fixed value may not hold.

Gravitational interactions between a hot Jupiter and its host star influence the planet’s orbit over time. However, the mechanisms governing these interactions—such as angular momentum transfer, energy dissipation, and internal stellar and planetary processes—are still not fully understood, making this an active area of research. By improving our understanding of these internal mechanisms, we can refine theoretical models of orbital decay, ultimately aiding in the identification of decaying systems and potential merger events.

% ————————————————————————————————————————————————————————————————
% —————————————————————    NEW SECTION    —————————————————————
% ————————————————————————————————————————————————————————————————

\subsection{Observational Evidence and Challenges}\label{sec:obs}

When a planet transits its star, it can be observed using both photometric and spectroscopic techniques, each producing distinct datasets and revealing different planetary parameters. While both methods confirm the presence of an exoplanet, they provide complementary insights into its physical and atmospheric properties. Below is an overview of these observational techniques, the parameters they can help us determine, and the challenges associated with data collection and long-term maintenance.

\subsubsection{Photometric Techniques}\label{sec:obs-photo}

Photometric techniques involve measuring the intensity of a star’s light (flux) over time. These observations are commonly made with telescopes across various wavelengths, primarily in the optical and near-infrared, which we focus on in this section.

As reviewed in Section \ref{sec:intro}, a transit occurs when an exoplanet passes in front of its host star from our vantage point, blocking a portion of the star’s light. By collecting stellar flux measurements over time and plotting them as a light curve, we observe a characteristic dip in brightness as the planet transits. See Figure \ref{fig:winn-fig8} for an example of a light curve.

UHJs are particularly well-suited for photometric observations. Their large radii—often further inflated by intense stellar heating and tidal interactions \citep{ogilvie2014tidal}—cause deeper transit dips in light curves, making them easier to detect with lower uncertainty. Additionally, their short orbital periods result in frequent transits, allowing multiple observations within a short timeframe. For instance, an exoplanet with a three-day orbit could produce up to ten transits in a single month of observation.

The combination of large radii and rapid orbits also enables the detection of secondary eclipses, or occultations. When the planet orbits behind its star, the system’s total brightness momentarily decreases as the planet’s dayside (which also reflects light) is blocked. Though the occultation signal is weaker than the transit signal, it provides valuable constraints on the planet’s orbit. While most exoplanets are too small to produce a detectable occultation, UHJs offer rare opportunities to observe these events in light curves.

\begin{figure}[htbp]
    \centering
    \includegraphics[width=0.6\linewidth]{figs/winn_fig2.png}
    \caption{Figure 2 from \citet{winn2010transits} illustrates the four key contact points during an exoplanet transit. First contact ($t_I$) marks the beginning of the transit when the planetary and stellar disks first touch. Second contact ($t_{II}$) occurs when the planet is fully within the stellar disk, defining the ingress duration as $\tau_\text{ing} = t_{II} - t_I$. As the planet moves across the stellar disk, it reaches third contact ($t_{III}$), the moment it begins to exit. Fourth contact ($t_{IV}$) marks the end of the transit when the disks last touch, defining the egress duration as $\tau_\text{egr} = t_{IV} - t_{III}$. The total transit duration is given by $T_\text{tot} = t_{IV} - t_{I}$, spanning the entire event. The full transit duration, $T_\text{full} = t_{III} - t_{II}$, represents the time the planet is completely within the stellar disk.}
    \label{fig:winnfig2}
\end{figure}

UHJs produce well-defined light curves, but how do we extract meaningful information from them? One key measurement is the midtransit time—the moment when the planet is at the center of its transit. This is calculated as half the total transit duration, or 
$\frac{T_{\text{tot}}}{2}$, as illustrated in Figure \ref{fig:winnfig2}.

The interval between consecutive midtransit times corresponds to the planet’s orbital period, the time it takes to complete one full orbit around its star. By continuously monitoring UHJs and measuring their midtransit times, we can track how their orbital periods evolve over time.

If the orbital period remains consistent, we infer that the UHJ's orbit is stable. However, a gradual change—or more critically, a decrease—in the orbital period suggests that the planet’s orbit is shrinking. This decay is likely driven by tidal interactions with its host star, which, if sustained, will eventually lead to the planet’s inspiral and ultimate merger with the star \citep{ogilvie2014tidal}.

\begin{figure}[htbp]
    \centering
    \includegraphics[width=0.8\linewidth]{figs/precession.jpg}
    \caption{Adapted from \citet{vervoort2022system}, this illustration depicts apsidal precession. The left panel shows an exoplanet initially orbiting within a fixed reference plane. Over time (right panel), the planet’s orbital ellipse precesses, gradually rotating within its plane. When observed edge-on, this precession causes variations in transit mid-times, making it appear as though the orbital period is changing. Without careful analysis, such shifts could be misinterpreted as evidence of tidal decay.}
    \label{fig:precession}
\end{figure}

Changes in the orbital period may not always indicate tidal decay. The orbital period may also change due to apsidal precession, which occurs when a planet's orbit rotates within its orbital plane, illustrated in Figure \ref{fig:precession}. This evolution can cause the time between transits to vary sinusoidally. However, precession also shifts the timing of occultations by an amount of time equivalent to but opposite in sign from the shift in transit timing. In contrast, if a planet's transit timing changes due to orbital decay, then the times between occultations decrease along with the times between transits.

The ability to track transit and occultation timing variations in UHJs provides a powerful tool for studying orbital evolution. By carefully distinguishing between true orbital decay and alternative effects like apsidal precession, we can refine our understanding of the mechanisms driving planetary migration and eventual inspiral. As photometric monitoring techniques continue to improve, long-term observations of UHJs will be crucial for confirming tidal decay candidates and constraining theoretical models. In the next section, we explore how spectroscopic techniques complement photometric observations by providing additional insights into orbital dynamics.

\subsubsection{Spectroscopic Techniques}\label{sec:obs-spec}

Spectroscopy is the study of how matter interacts with electromagnetic radiation, specifically through the absorption and emission of light at different wavelengths. When applied to exoplanet observations, spectroscopy during transit or occultation events provides valuable information about a planet’s motion and position within its orbit. While spectroscopy is also often used to analyze the chemical composition of planetary and stellar atmospheres, this review focuses on its role in determining orbital dynamics and transit characteristics. Spectroscopy of exoplanets is commonly taken through radial velocity (RV) measurements, which track the motion of the star along the line of sight by detecting shifts in the stellar spectrum \citep{lovis2010radial}. In this section, we will explore the Rossiter-McLaughlin effect, which provides additional information about a star's orbit.

% ————The Rossiter-McLaughlin Effect————
% WHAT IS IT
The Rossiter–McLaughlin (RM) effect, first described in the context of eclipsing binary stars by \citet{rossiter1924detection} and \citet{mclaughlin1924some}, and later observed in exoplanets by \citet{queloz2000detection} and \citet{bundy2000search}, is a direct consequence of the Doppler effect. The Doppler effect refers to the change in frequency (or wavelength) of a wave as the source moves relative to the observer. A familiar example of this is the change in pitch of a siren as an emergency vehicle approaches and then moves away. As the vehicle approaches, the sound waves in front of it are compressed, causing the pitch to increase (higher frequency). Conversely, as the vehicle retreats, the sound waves behind it are stretched, lowering the pitch (lower frequency).

The same principle applies to light waves. When a light source moves toward an observer, the wavelengths of light are compressed, resulting in a shift towards the blue end of the electromagnetic spectrum, known as a blueshift. Conversely, when the light source moves away from the observer, the wavelengths are stretched, causing a shift towards the red end of the spectrum, referred to as a redshift.

% \begin{figure}[htbp]
%     \centering
%     \includegraphics[width=0.7\linewidth]{figs/gaudi_fig1.jpg}
%     \caption{Adapted from Figure 1 of \citep{gaudi2007prospects}, this illustration demonstrates the physics of the Rossiter-McLaughlin effect across the different stages of an exoplanet transit. The first row shows the blue and red-shifted hemispheres of the star, highlighting the color-shifted starlight that is partially blocked by the planet during the transit. The second row demonstrates the characteristic "bump" in a stellar absorption line that occurs due to the Doppler shift in frequency as the planet transits the star, creating a distinct signature in the spectral data.}
%     \label{fig:gaudi-fig1}
% \end{figure}

\begin{figure}[htbp]
    \centering
    \includegraphics[width=0.7\linewidth]{figs/triaud_fig1.png}
    \caption{Figure 1 from \citep{triaud2017rossiter} illustrates the Rossiter-McLaughlin effect for the hot Jupiter HD 189733A b, as observed through RV measurements. A visual of the exoplanet transit is provided in the bottom left corner. The RV measurements are color-coded with their Doppler shift to convey the passage of the exoplanet across the stellar surface. Parameters obtained from this observation, including the stellar surface rotation speed $v \sin i_*$, the transit depth $D$, and the sky-project spin-orbit angle $\lambda$, are included in the top right.}
    \label{fig:triaud-fig1}
\end{figure}

When an object as large as a star rotates, the hemisphere moving toward the observer is blueshifted, while the hemisphere moving away is redshifted, as shown in the bottom left corner of Figure \ref{fig:triaud-fig1}. 

As a planet transits the star, it sequentially blocks portions of the star with different Doppler shifts, resulting in characteristic patterns in spectroscopic data. During RV measurements, the transiting planet will cover the redshifted hemisphere or the blueshifted hemisphere at different points in the transit. When the planet covers the blueshifted hemisphere (the portion of the star moving toward us), it blocks the light that would otherwise be blueshifted, causing the overall flux to shift redward, resulting in a positive shift in RV measurements. Conversely, when the planet covers the redshifted hemisphere (the portion of the star moving away from us), it blocks the light that would otherwise be redshifted, causing the overall flux to shift blueward, leading to a negative shift in RV measurements. These characteristic "bumps" in the RV curve during a transit are illustrated in Figure \ref{fig:triaud-fig1}.

% WHAT CAN WE FIND
\begin{figure}[htbp]
    \centering
    \includegraphics[width=0.9\linewidth]{figs/winn_fig6.jpg}
    \caption{Figure 6 from \citet{winn2010transits} illustrates the measurement of the spin-orbit angle $\lambda$ using the RM effect for three transit scenarios. While all systems share the same impact parameter ($b$), and thus produce identical photometric observations, they differ in their spin-orbit angle ($\lambda$). As a result, each scenario exhibits a unique RV curve. The dashed line represents the RM effect in the absence of a transit event, providing a baseline for comparison. The dotted lines illustrate the RM effect during a transit event assuming no limb darkening, while the solid lines account for limb darkening.}
    \label{fig:winn-fig6}
\end{figure}

The main two parameters we can derive from the RM effect are the projected rotation speed of the stellar surface ($v \sin i_*$) and the sky-projected angle between the planet's orbital path and the stellar equator, $\lambda$.

If $\lambda = 0$, this means that the planetary orbit and the stellar spin axis are aligned, as seen in the left panel of Figure \ref{fig:winn-fig6}. However, if $\lambda > 0$, there is a misalignment, and the planet's orbit is tilted relative to the stellar rotation, as seen in the middle and right panels of Figure \ref{fig:winn-fig6}.

This alignment (or lack thereof) has important implications for understanding the planet's orbital evolution. Combined with other planetary and stellar parameters, $\lambda$ can provide insights into tidal decay rates. As discussed in section \ref{sec:tidaldissipation}, tidal interactions between the planet and its host star tend to drive spin-orbit alignment over time. When the orbital and stellar spins are aligned ($\lambda = 0$), angular momentum transfer between the two bodies is most efficient, leading to faster orbital decay \citep{ogilvie2014tidal}. Thus, measuring $\lambda$ can offer valuable clues about the system’s long-term evolution.

Moreover, the spin-orbit alignment sheds light on the planet's migration history. Disk migration tends to result in co-planar orbits, with $\lambda \approx 0$, while non-coplanar orbits, typically associated with larger values of $\lambda$, are often a signature of high-eccentricity migration \citep{triaud2017rossiter}. Therefore, studying $\lambda$ not only enhances our understanding of tidal dissipation but also provides insight into the mechanisms that drive the migration of ultra-hot Jupiters.

\citet{attia2023dream} recently compiled spin-orbit angles of several hot Jupiters and found that aligned systems are dominant in their sample. However, they also observed that misaligned systems do not appear randomly distributed; instead, they tend to cluster around $90^\circ$ orbits. This pattern suggests that there may be another mechanism driving the excitation of spin-orbit angles, which has yet to be fully explored. The accumulation of misaligned orbits around $90^\circ$ could indicate that the orbits of hot Jupiters are sometimes excited, preventing efficient angular momentum transfer and stalling the decay of their orbits. Further investigation is required to determine whether this excitation is tied to specific conditions or physical processes that might explain the lack of observed orbital decay in certain systems. Such research could provide new insights into the dynamics of ultra-hot Jupiter systems and the mechanisms that influence their orbital evolution.

Collecting RV measurements using spectrographs is crucial for understanding both the past and future evolution of hot Jupiters. While this field has already seen significant progress and observations are at an all-time high, the continued increase in observations through new missions and advanced spectrographs, when combined with photometric data, will be invaluable.


\subsubsection{Ground vs. Space Based Observations}

With many new space telescope missions planned for the future, along with the increasing usage of ground-based observatories and personal telescopes, the opportunity for data collection has never been greater. Both photometric and spectroscopic observations can be made on the ground and in space, but each type of observation has its advantages and challenges.

Some observatories, both space and ground-based, are limited in their observation capabilities, typically collecting data from only one target at a time. These observatories often require users to submit requests for data, making access more challenging, especially for popular missions like the James Webb Space Telescope (JWST).

On the other hand, space-based missions like NASA's Transiting Exoplanet Survey Satellite (TESS) continually observe different sectors of the sky. TESS collects stellar flux data from a different part of the sky about every month, making it an invaluable resource for exoplanet research. Future ground-based missions, such as the Vera C. Rubin Observatory, will scan the sky nightly with the largest camera ever built, generating massive amounts of data. While this data will be publicly available, processing and storing such an enormous volume of information will present significant challenges, likely requiring advanced data processing pipelines.

There is also a rise in the use of personal telescopes, often referred to as amateur or citizen astronomy. For example, the collaboration between Unistellar and the Search for Extraterrestrial Intelligence (SETI) Institute has supported citizen astronomers in collecting transit observations and contributing them to publicly accessible databases \citep{peluso2023unistellar}. The recent confirmation of the TESS object of interest TIC 393818343 b through citizen astronomer observations made with Unistellar telescopes \citep{sgro2024confirmation} demonstrates the value of such data. This is particularly true for studying the orbital evolution of hot Jupiters, as long-term observations are necessary to detect the subtle orbital changes indicative of tidal decay.

\begin{figure}[htbp]
    \centering
    \includegraphics[width=0.5\linewidth]{figs/winn_fig8.png}
    \caption{Comparison of ground-based observations (left) vs. space-based observations (right) of transit light curves.}
    \label{fig:winn-fig8}
\end{figure}

While ground-based observations are becoming increasingly popular, especially with support from citizen scientists, challenges remain. Earth's atmosphere can introduce noise into photometric observations, especially in the presence of clouds or atmospheric pollution. Additionally, the deflection or absorption of certain wavelengths by the atmosphere can make ground-based spectroscopic observations nearly impossible. Figure \ref{fig:winn-fig8} shows a comparison between ground-based and space-based transit light curves. Although the general patterns are similar, precise measurements of transits and midtimes are necessary to track orbital changes, especially for studying tidal decay. While it is entirely possible to use ground-based observations to track orbital changes, more precise measurements help mitigate errors and improve data interpretation.

Additionally, some systems have extremely dim stars or very small transit signals that cannot be detected well enough by ground-based observatories. These systems, such as Kepler-1658 b, show historical tendencies towards possible orbital evolution, but require observation by space-based observatories \citep{adams2024doomed}. Some of these issues are explored further in the following subsection.

Both ground- and space-based observations are invaluable for exoplanet research. Space-based observations, while harder to access or more challenging in terms of data processing, offer high precision, making them ideal for long-term studies of orbital periods in ultra-hot Jupiter (UHJ) systems. However, the growing community of citizen scientists has dramatically increased the volume of ground-based observations. While maintaining consistency in observations is crucial, the trade-off between the quality and quantity of data is discussed in Section \ref{sec:models}. Ultimately, both types of observations contribute significantly to our understanding of exoplanetary systems.

\subsubsection{Challenges in Maintaining Proper Data Collection Practices}

% With increasing observations over time, many now reaching the decade mark, it has become easier to identify trends in the data. However, some systems with limited observations over the years often require additional data to rule out various phenomena. It is crucial to account for tidal decay by considering alternative explanations, such as apsidal precession, the effects of companion objects, and line-of-sight acceleration.

With observations of hot Jupiters increasing over time, many now even reaching the decade mark, it has become easier to identify trends in the midtime data indicative of tidal decay. Because changes in orbital period are very small (on the scale of milli-second per Earth year), detecting timing signatures in exoplanet systems demands highly precise timing analysis. Several issues in the timing data can arise in this process, including duplicate times, composite times, and false timing conversions \citep{adams2024doomed}. 

Duplicate times occur when the same observation is recorded twice in a dataset, making the observation at this point twice as significant when processed. These can significantly impact data accuracy, especially when they occur at critical points or when errors are small. These errors are easily avoidable with proper data maintenance. 

To mitigate errors caused by the Earth's motion around the Sun and the Sun's motion around the solar system barycenter, a dynamical timing system is necessary to avoid cyclical variations in astronomical time series. This is why we use the Barycentric Julian Date (BJD) TDB timing system and scale. Historically, the Heliocentric Julian Date (HJD) was more commonly used because it is easier to calculate. Although one could argue that using HJD or BJD does not significantly impact most observations—since the difference is about one second and most transit observations do not have the accuracy to be affected—there is an important distinction between the two. HJD is typically based on the UTC time scale, while BJD is based on the TDB time scale, and the difference between these scales is approximately 64.184 seconds (as of 2024), with this offset changing as leap seconds are added. For this reason, it is crucial to report both the timing system and time scale used in any data analysis, as they are not always explicitly stated and are often assumed to be associated (UTC with HJD, TDB with BJD).

Composite times, which are derived from multiple light curves, also present several challenges. The error bars on composite times are typically much smaller because of this consolidation of timing information into a single point. Smaller error bars skew model fits (see Section \ref{sec:models} for further explanation of this process), resulting in possible false detections of decreasing orbital periods. Composite times can also obscure the original data source, making it harder to identify potential timing errors and may lead to double-counting transits.

Composites are often used in cases such as stacked light curves, where transits from multiple observations are combined to reduce noise and improve detection. This is common in ground-based surveys where detecting a single transit is difficult. Another example is when midtimes are reported for composite data, anchoring all combined transits into one observation. This approach results in a specific (and usually small) error for the mid-time. Weighted average times are another example, where two mid-times from different sources are averaged together. This makes it challenging to deconstruct the data and identify potential errors in the observations. An example of how composite mid-times can lead to issues is seen with CoRoT-2, where 82 transits were stacked. When unstacked, the original transits had errors that were 50\% larger, changing the delta BIC from 33 to -2, effectively ruling out tidal decay \citep{adams2024doomed}.

<TODO: Write about uncertainties on timing data>

To combat these problems, it is essential to identify the exact timing system used to prevent false conversions and ensure the original source of the data is clearly specified to avoid duplicates.

% - With increasing observations over time (a lot now reaching the decade mark) it is easier to start learning trends in the data. However, some systems with limited observations over the years usually are ruled out with additional observations. It is also important to rule out tidal decay by looking at all the other options (apsidal precession, effects of companion objects, line of sight acceleration).

% - Detection of these timing signatures requires very precise timing analysis. Some issues that have come up with this are detecting duplicate times, composite times, and false timing conversion. This is why it is VERY important to identify exact timing system (to combat false timing conversion) and exact source of the original data (to combat duplicates). 

% - Duplicates can have a large impact if it has low errors or occurs at critical point. They are source of error that can be completely eliminated with proper data maintenance.

% - Composites (times derived from more than 1 light curve) can have the following issues: their error bars are usually much smaller because they compress all timing info into single point, they obscure original data source which makes it difficult to find potential timing errors, and can lead to double counting transits

%      - Composited include the following: stacked light curves where transits from multiples epochs are stacked to actually see the transits, which is common in ground-based surveys where it is difficult to detect a single transit; when ephemeris mid-times are reported for composite data, this usually anchors all the combined transits into one reported epoch, which results in a very specific (small error) mid-time for one epoch and will weight the BIC equation; weighted average times are when two mid-times from different sources are averaged together, which makes it difficult to deconstruct this and look for errors in the observations
     
%      - Example of a composite mid-time is for CoRoT-2 where 82 transits were stacked and when unstacked, the original transits had 50 percent larger errors, which changed delta BIC from 33 to -2 (pretty much ruling out tidal decay)
     
% - Timing systems: 

%     - to remove the motion of the earth around the sun and the motion of the sun around the solar system barycenter, we must use a dynamical timing system to avoid cyclical variations in astronomical time series. This is why we use BJD TDB timing system and scale (respectively). 
    
%     - Historically, helicentric julian date HJD is much easier to calculate and this has been preferred over BJD, and one can argue that using one or the other doesn't have much significance as the difference is only about 1 second and most transit observations do not have the accuracy within this range to be affected. However, HJD is usually in UTC time scale while BJD is usually in TDB timescale. This difference is about 64.184 seconds as of 2024 (this offset changes with addition of leap seconds). This is why it is very important to also report the time SCALE as well as the system, which isn't always reported and more often than not assumed to be associated (UTC with HJD, TDB with BJD). 

% - Running models: fit to linear and quadratic models, calculated delta BIC, ran omit-one delta BIC test, calculated a rescaled delta BIC, investigated outliers that failed the omit one test and re-ran everything again with corrected data
% - Timing Analysis: 
%     - WASP-12 b: still looking good
%     - WASP-19 b: period of just 0.79 days and data spanning 15 yrs (good candidate), however, decaying at an order of magnitude smaller than 12b which makes Q*' an order of magnitude larger, however, timing errors that were corrected made the delta BIC drop to insignificant levels, this may be ruled out as candidate
%     - CoRoT-2 b: Data spanning 18 yrs, had composite errors (in CoRoT mission data, 82 transits were combined), timing scale errors (in ETD, data was labeled as HJD when it was really JD and in a Ozturk&Erdem, data was labeled as BJD/TDB but was possibly JD), and duplicate errors. Once all of these errors were addressed, the results delta BIC was -2, showing no tidal decay
%     - WASP-121 b: the early data really need to be refit but the actual timing data has not been published, so this cannot happen. When scaled, the delta BIC still stayed positive (14), showing that the error bars are good and this system just really needs more data observed
%     - WASP-46 b: Has good data, lots of scatter in data before 2015 with 7 outlier points, removing these points changes delta BIC from 32-149, and when rescaled changes to 2, flagged for reanalysis
%     - TrES-1 b: good baseline of data and no obvious timing errors, has siginifgant positive delta BIC (69) HOWEVER, the Q*' value is five orders of magnitude lower than what we expect for the decay rate, so the decay is most likely caused by something else
%     - Kepler-1658 b: has a very shallow transit depth, so hard to observe with ground-based telescopes. rapidly decaying and star is evolved off MS, weinburg says g-mode does not explain rapid decay but vissapragada says could be inertial wave dissipation and rapid stellar rotation. confirmation could be possible with a few more observations, so withing a year or two


% - Answering the question: was WASP-12 b the first world to have detected tidal decay because it had enough transits observed OR is there something special about the WASP-12 b system?
%     - Test 1: Sets all planet's decay rates to WASP-12 b decay rate and sees what the delta BIC would be given the number of observations each system has. Shows that nearly half of the systems would have has delta BIC > 30 (in reality only 1 did) and 4 systems would have had 10x larger delta BICS than WASP-12b
%     - Test 2: Sets Q*' (tidal dissipation factor) to be the same for all systems and calculated the decay rate for each system. seven planets had value >= 3x the observed error in the decay rate
%     - Overall, for 20 of the systems, there is evidence that we should have detected tidal decay but we have not


% - CONCLUSIONS: 
%     - For the two systems where orbital decay is likely (WASP-12 b and Kepler-1658 b), their host stars may have evolved off the MS which would offer explanation for rapid decay and maybe point towards evidence that a star needs to be evolved off the MS to exhibit decay (QUESTION: but then what does that say about the observation that most UHJ are around young stars?)
%     - delta BIC values below 30-50 should be tentative because of usual timing data errors, especially in the estimation of timing uncertainties
%     - systems need to be observed at least once a year

\begin{figure}[htbp]
    \centering
    \includegraphics[width=0.9\linewidth]{figs/adams_fig2.jpg}
    \caption{$\Delta$BIC values for 43 exoplanets. In all panels, a positive $\Delta$BIC value favors a quadratic model over a linear model, indicating a change in the orbital period. Blue bars represent planets with a negative $\dot{P}$ (decreasing orbital period), while red bars indicate a positive $\dot{P}$ (increasing orbital period). Two reference lines are included: a dashed line at $\Delta$BIC = 30 and a dotted line at $\Delta$BIC = 10. 
    The top panel displays the full range of $\Delta$BIC values, with WASP-12 b standing out as a clear outlier. The middle panel zooms in to highlight the smaller $\Delta$BIC values, showing the five systems with the next highest $\Delta$BIC values after WASP-12 b. For these systems, uncertainties on their measured orbital period changes (P???) are plotted below the corresponding bars. The bottom panel presents the $\Delta$BIC values after applying a rescaling procedure to the data. Details of this rescaling process are discussed below.}
    \label{fig:adams-fig2}
\end{figure}

For these planets, we are particularly interested in the decreasing orbital periods when searching for tidal decay. Increasing orbital periods, while interesting, do not indicate tidal decay and therefore are not of interest to us in this instance. 
% Note: before, this was then followed by a discussion on delta BIC, so will have to find a way to transition again

% Describing the rescaling process
In reported literature, uncertainties on mid-transit times in light curves can sometimes be unrealistically small. When fitting a model to transit data, each data point is weighted according to its photometric uncertainty, $\sigma$, with weights proportional to $\frac{1}{\sigma}$. If a data point has an exceptionally small uncertainty, it disproportionately influences the model fit, effectively "weighing down" the resulting model and skewing the fit toward that point.

To mitigate this issue, \citet{adams2024doomed} implemented a rescaling procedure. They first performed a linear fit using the original timing uncertainties and calculated the corresponding $\chi^2$ value. Then, they uniformly scaled all uncertainties by a factor such that the resulting $\chi^2$ value equaled 1. After rescaling, \citep{adams2024doomed} recomputed the $\Delta$BIC values for all systems, as shown in the bottom panel of \ref{fig:adams-fig2}. This adjustment significantly altered the $\Delta$BIC values, with only one system—excluding WASP-12 b—retaining a $\Delta$BIC greater than 10.

This change in $\Delta$BIC highlights the importance of data processing and accuracy. 

% TODO: Add a plot here to show how unrealistic transit midtime uncertainties can affect the resulting model fit.

\begin{figure}[htbp]
    \centering
    \includegraphics[width=\linewidth]{figs/adams_fig4.jpg}
    \caption{Caption}
    \label{fig:adams-fig4}
\end{figure}

In \ref{fig:adams-fig4}, the data points highlighted by red diamonds have very small error bars compared to the other observations in the data set. These data points greatly affect the value of $\Delta$ BIC, shifting it by 25\% or more when included.

% \subsubsection{Case Study: WASP-12b}
% Discuss the observed orbital decay in WASP-12b, including the measured decrease in transit intervals and its implications. 

% \subsubsection{Long-Term Survey Findings}
% Review the findings of Adams et al. (2024), which reported no new evidence for orbital decay in a long-term survey of 43 UHJs.

% \subsubsection{Discrepancies and Challenges}
% Analyze the discrepancies between different observational studies and the challenges in detecting small changes in orbital periods.


% ————————————————————————————————————————————————————————————————
% —————————————————————    NEW SECTION    —————————————————————
% ————————————————————————————————————————————————————————————————


\subsection{Analytical Models \& Metrics for Optimizing Searches for Tidally Decaying Exoplanets}\label{sec:models}

\subsubsection{Analytical Models \& Proposed Metrics}
% Discuss the analytical approximations for transit light-curve observables presented by Carter et al. (2008).

Analytical models are useful in estimating the parameters of a system and predicting the behavior of the bodies in the system. 

As mentioned in the introduction, <quickly review exoplanet transits and occultations>. Transit and occultation light curves can tell us about planetary and stellar radii, orbital inclination, and mean density of the star. By tracking changes in mid-times and/or orbital parameters, we can also detect other planets in system.

Direct observations of transits and/or occultations of course are valuable, but we can also do some predictions on these observations to do things like predict the likelihood of a future transit, <something else here>, or even optimize future observing plans to maximize the likelihood of detecting orbital decay. 

Even when it is possible to have numerical values, it can help to have analytical approximations as well as uncertainties and covariances. 

Analytical approximations are good for planning observations. For example, which systems will have the most precise measurement of orbital inclination, how many transits do I need to observe before statistical error in planetary radius is smaller than systematic error. They can also help in speeding up the convergence of optimization algorithms, and can be useful in order-of-magnitude estimates of observability of subtle transit effects such as transit-timing variations (TTVs), precession-induced changes, and asymmetry in ingress and egress due to non-zero eccentricity.

\begin{figure}[htbp]
    \centering
    \includegraphics[width=0.8\linewidth]{figs/carter_fig1.png}
    \caption{The piecewise linear model (solid) and exact source model (dashed) of a transit light curve. The vertical axis shows normalized flux, with constant starlight described as $f_0$ and $f_0-\delta$ showing the maximum amount of light being blocked by a star. The horizontal axis shows the four contact times, as also described in Figure \ref{fig:winnfig2}, with the transit mid-time denoted by $t_c$. The fractional transit depth is denoted by $\delta$, T is approximately the transit duration, and $\tau$ is approximately the ingress/egress duration.}
    \label{fig:carter-fig1}
\end{figure}

% % Carter 2008: Analytical Approximations for Transit Light-Curve Observables, Uncertainties, and Covariances 
% \cite{carter2008analytic}

% - Their models assume flux measurements are made continuously throughout transit and stellar limb darkening is neglible
% - Approximation of no limb darkening can ACTUALLY be used for mid-IR bandpasses (WHY??) so everything in this paper assumes observations around the IR bandpasses so that limb darkening effects can be ignored
% - Equation 1 and 2 are exact solutions, to make more analytical approximations, some assumptions are made:
%     - assume orbital period is large compared to transit duration
%     - assumes transit occurs in a uniform, straight line motion across stellar disk (meaning it has constant velocity, QUESTION: what would it look like if it didn't? is this something worth discussing?)
% - Equation 3 is analytical approximation given the above assumptions
% - Equations 4 and 5 are analytical approximations of two characteristic timescales, time between ingress and egress midpoints and total "ingress" time
% - These equations can be expanded to include eccentric orbits by replacing a (semimajor axis) and n (angular frequency)
% - Equations 8-11 are for a piecewise "linear in time" model light curve denoted as F1
%     - Deviation in this approximation occur near ingress and egress phases (QUESTION: is this because this is linear piecewise function?)
%     - Most accurate for small values of r & b
%     - Least accurate for grazing transits
% - Uses Fisher information analysis to find correlation between parameters (there are 5 in the model: t_c (time of conjunction), tau (QUESTION: what exactly is tau on pg 2), T (total transit time?), delta (idk), and f_0 (unocculted flux)
%     - When you look at partial time derivative of each parameter, you can see that the model reflects the symmetry of the "real" behavior
%     - t_c is uncorrelated with the other params
% - Equation 20 is the covariance matrix for the piecewise function
% - Equation 21 is the covariance matrix for the piecewise function if we assume we have a great baseline of the unocculted flux f_0 so that f_0=1, making delta the fractional transit depth
%     - From this we can see that theta is the key controlling param, where theta = r/(1-b^2)

\begin{figure}[htbp]
    \centering
    \includegraphics[width=0.5\linewidth]{figs/carter_fig2.png}
    \caption{Caption}
    \label{fig:carter-fig2}
\end{figure}

\begin{figure}[htbp]
    \centering
    \includegraphics[width=0.5\linewidth]{figs/winn_fig3.png}
    \caption{Geometry of a transit and the likelihood of the transit being seen (can tie into the analytical models from Carter or just use it to explain why we would want an idea of the likelihood (because it may not happen based on complicated orbital properties)).}
    \label{fig:winn-fig3}
\end{figure}

% Describing delta BIC
When fitting a model, we determine the accuracy of the model fit to the actual data using the Bayesian Information Criterion (BIC) \citep{schwarz1978estimating}

\begin{equation}
    \text{BIC} = \chi^{2} + k\text{ln}N
\end{equation}

where $k$ is the number of fit parameters (two for a linear model, three for a quadratic model, and five for a precession model), $N$ is the number of data points or observations used to fit the model, and $\chi^{2}$ is the reduced chi-squared, which is the measurement of the goodness of fit, calculated by the sum of the residuals squared \citep{pearson1900criterion}:

\begin{equation}
    \chi^{2} = \sum \left( \frac{O_i - E_i}{\sigma_i} \right)^2
\end{equation}

where $O_i$ is the actual observed value, $E_i$ is the expected value calculated by the model, and $\sigma_i$ is the photometric uncertainty.

Smaller values of $\chi^2$ indicate smaller residuals, suggesting a better fit of the model to the data. However, models with more fit parameters can become biased, fitting the data too closely. For example, in Figure \ref{fig:reduced-chi-squared}, we compare three models fitted to synthetically generated data: a linear model with two parameters, a quadratic model with three parameters, and a sinusoidal model with five parameters. As the number of fit parameters increases, the $\chi^2$ value also rises, suggesting that more complex models may better explain the data pattern. However, when we account for the penalization of additional parameters using BIC, the value of BIC decreases, showing that more complex models do not necessarily provide a better description of the data. To minimize bias in more complex models, we use BIC, which adds a penalty based on the number of fit parameters $k$ to the reduced $\chi^2$ value.

\begin{figure}[htbp]
    \centering
    \includegraphics[width=\linewidth]{figs/reduced_chi_squared.png}
    \caption{Comparison of model fit analysis to synthetic data. Residuals, represented by the dashed lines, show the differences between the actual data (black circles) and the model fits (solid lines). The reduced $\chi^2$ and BIC values for each model are displayed above the corresponding plot.}
    \label{fig:reduced-chi-squared}
\end{figure}

In this case, positive values of $\Delta$ BIC, calculated with the equation:

\begin{equation}
    \Delta \text{BIC} = \text{BIC}_{\text{lin}} - \text{BIC}_{\text{quad}}
\end{equation}

indicate a statistical preference for a quadratic model over a linear model.

\subsubsection{Implications for Detecting Orbital Decay}
% Examine how these models assist in accurately determining transit times and detecting minute changes indicative of orbital decay.

\begin{figure}[htbp]
    \centering
    \includegraphics[width=\linewidth]{figs/jackson_fig1.png}
    \caption{The uncertainty on the mid-transit time for multiple extrasolar systems based on the simplified transit model from \citet{carter2008analytic}, normalized to WASP-12 b's mid-time uncertainty, versus the change in orbital period with respect to time.}
    \label{fig:jackson-fig1}
\end{figure}

The y-axis of Figure \ref{fig:jackson-fig1} in \citet{jackson2023metrics} shows the analytical uncertainties of transit mid-times using the equation:

\begin{equation}
    \sigma_{t_c} = \sqrt{\frac{\tau}{2\Gamma}} \left(\frac{\sigma}{\Delta}\right) = \sigma \Gamma^{-1/2}
    \times \left( \frac{R_\star}{a} \right)^{1/2} \left( \frac{P}{2\pi} \right)^{1/2} 
    \left( \frac{R_p}{R_\star} \right)^{-3/2} (1 - b^2)^{-1/4} 
    \times \begin{cases}
    1, & \text{transit} \\
    \left(\frac{I_p}{I_\star}\right)^{-1}, & \text{eclipse}.
    \end{cases}
    \label{eq:analytical_transit_uncertainties}
\end{equation}

where $\tau$ is the ingress/egress duration, $\Gamma$ is the sampling rate for the transit observations (assumed constant), $\sigma$ is the per-point photometric uncertainty, $\Delta$ is the transit or eclipse depth, $R_{*}$ is the stellar radius, $a$ is the orbital semimajor axis, $P$ is the orbital period, $R_{p}$ is the planetary radius, $b$ is the impact parameter, and $I_{p/*}$ is the planetary/stellar disk-integrated intensity. 

For comparison, uncertainties are normalized to the transit mid-time uncertainty of WASP-12 b, which is one of the only confirmed tidally decaying exoplanets. 

The x-axis of \ref{fig:jackson-fig1} shows the analytical orbital decay rate, defined as the change in orbital period over each consecutive orbit, also called an epoch, using the equation: 

\begin{equation}
    \frac{dP}{dE} = P \frac{dP}{dt} \approx - (26 \, \mu s \text{ per orbit})
    \times \left( \frac{M_p}{M_{\text{Jup}}} \right)
    \left( \frac{M_\star}{M_\odot} \right)^{-8/3}
    \left( \frac{R_\star}{R_\odot} \right)^{5}
    \times \left( \frac{P}{\text{day}} \right)^{-10/3}
    \left( \frac{Q_\star}{10^5} \right)^{-1},
    \label{eq:orbital_period_rate_of_change}
\end{equation}

where $\frac{dP}{dt}$ is the change in orbital period with respect to time, $M_p$ is the planetary mass in Jupiter masses ($M_{Jup}$=
1.89813 × 1027 kg), $M_*$ is the stellar mass in solar masses
($M_e$ = 1.989 × 1030 kg), $R_*$ is the stellar radius in solar radii ($R_e$ = 6.957 × 108 m), $P$ is the orbital period in days, and $Q_*$ is the star’s modified tidal dissipation parameter.

Figure \ref{fig:jackson-fig1} shows an upwards trend in this relationship, meaning as the rate of change of the planet's orbit increases, so does the uncertainty on its mid-transit time. So while systems with quickly changing orbital periods may be more interesting to focus on, the mid-time may be harder to predict, especially as time goes on. One thing to keep in mind with this figure is that the value of $\frac{dP}{dE}$ is based on equation \ref{eq:orbital_period_rate_of_change} with appropriate system parameters filled in, but assuming a WASP-12 b-like tidal dissipation parameter $Q_* = 2 \times 10^5$. However, in \citet{ogilvie2014tidal}, it is shown that the tidal dissipation parameter depends on stellar internal structure, which give rise to different spherical harmonics. Incorporating non-linear dissipation into theoretical models, \citet{weinberg2023orbital} found that the tidal dissipation parameter varies with stellar mass and age. Therefore, the value of $Q_*$ cannot be assumed to be constant for different extrasolar systems. For future work, an interpolable grid of $Q_*$ values would need to be generated based on stellar age and mass, and analytical approaches would need to incorporate a more refined estimate of the tidal dissipation parameter based on the stellar properties.

\begin{figure}[htbp]
    \centering
    \includegraphics[width=\linewidth]{figs/jackson_fig2.png}
    \caption{(Top) The blue dots show the uncertainty on the predicted transit mid-time of WASP-12 b as observations increase. The orange line shows the value of $\Delta$BIC as observations increase. (Bottom) The blue dots show the observed transit mid-time with the linear ephemeris subtracted to show the trend in non-linear parameters (the orbital decay rate $\frac{dP}{dE}$). The orange line shows the quadratic ephemeris term $\frac{dP}{dE}$ = -86.7 $\mu$s orbit$^{-1}$, which is pulled from the orbital decay model in \citet{yee2019orbit}.}
    \label{fig:jackson_fig2}
\end{figure}

Figure 2 in \citet{jackson2023metrics} shows how different analytical approximations may change as observations increase using the example of WASP-12 b data pulled from \citet{yee2019orbit}. The left x-axis of the top figure shows the uncertainty of the predicted transit mid-time. The right x-axis of the top figure shows the value of $\Delta$BIC comparing a linear ephemeris to a quadratic ephemeris. The x-axis of the bottom figure shows the difference between the observed transit mid-time and the linear ephemeris fit in seconds. The y-axis for both figures uses epoch to denote consecutive orbits, with the associated year shown at the top to aid in understanding the passage of time between observations. 

As the system is observed over time, the quadratic term of the linear-subtracted mid-times becomes more apparent. However, in the initial observations, the trend may appear linear. 

As observations increase for the system, the uncertainty on the predicted transit mid-time decreases and the value of $\Delta$BIC increases, showing preference for a quadratic model. However, these observations are fairly consistent, with more observations added about every year. If enough time has passed between observations such that the uncertainty on the mid-time surpasses the transit duration, then it may be difficult to predict future transit mid-times. Such systems are ideal for follow-up observations. Figure \ref{fig:jackson_fig3} shows the amount of time needed between observations before the transit duration surpasses the predicted error on the mid-time given by the following equation:

\begin{equation}
    t_{wait} = \sqrt{\sigma^{2}_{t^{pred}_{tra}} - \sigma^{2}_{T_{0}}} \left(\frac{P}{\sigma_{P}} \right)
    \label{t_wait}
\end{equation}

versus the time since the last recorded observation as of April 5th, 2023. The labeled systems are close to or have surpassed their $t_{wait}$ value and need follow-up observations to confirm their predicted transit mid-time is what we expect (and if it is not what we expect, then this may indicate changes in the orbital period). 

\begin{figure}[htbp]
    \centering 
    \includegraphics[width=\linewidth]{figs/jackson_fig3.png}
    \caption{Time needed to pass between observations for the uncertainty on the transit mid-time to surpass the transit duration ($t_{wait}$) versus the time that has passed since the first recorded transit ($T_0$) (generated April 5th, 2023). QUESTION: Is $T_0$ the first ever recorded transit or the time of conjunction or the last observed transit idk??????}
    \label{fig:jackson_fig3}
\end{figure}

Tidally Decaying Exoplanets
% \cite{jackson2023metrics}

% - For host stars rotating more slowly than the planet, interaction between planet and tidal bulge transfers angular momentum from the planet's orbit to the star (this can tie into OGLIVIE PAPER)
% - Stellar dissipation rate may be related to stellar luminosity
% - Once gas giant spirals into roche limit, tidal disruption can occur
%     - Stable: decay timescale determined by tidal decay rate
%     - UNstable: disruption proceeds rapidly
%     - Direct accretion if roche limit lies within star
% - Signs of tidal disruption: tidal/accretion-induced spin-up, young MS stars, anomalous chem signatures in red giants that may be caused by planetary engulfment (maybe in MS stars as well)
% - based on Q*' (which this scales with it), planetary engulfment occurs between 0.1 and 1/year in milky way HOWEVER, the value of Q*' may not be set and may depend on a few things (CAN TIE THIS INTO OTHER PAPERS), ONE WAY to constrain this observationally (and not theoretically with stellar models) is to observe tidal decay
% - Based on Carter 2008 equations, we get equation 11, the uncertainty on the central time (QUESTION: is this mid-time?), which shows that the uncertainty increases with photometric uncertainty, decreases as transit depth and sampling rate increase. Also increases with ingress/egress duration because if b->1 (very long ingress/egress) then the light curve would be nearly v-shaped and we would need the exact time it switches from a downwards trend to an upwards trend to get the mid-time accurately
% - Figure 1 shows possible mid-time uncertainties on some systems based on equation 12, which relates uncertainty to stellar magnitude
% - We can use equation 18 to calculate the time at which the uncertainty on the transit becomes larger than the transit duration. Figure 3 shows some systems with their critical point of the uncertainty growing larger than transit duration and time that has passed since its last observation. Anything below the y=x line is something that has maybe passed the point of uncertainty being larger than transit duration and thus would be useful to follow up on
% - Equation 22 predicts the orbital decay rate (dP/dE) based on stellar and planetary parameters. HOWEVER, applications in the paper use a constant fixed value of Q*, which isn't exactly true (can tie into Ogilvie paper)
% - Note: in figure 4, there is a S/N which means signal to noise, which is just the "signal" or the actual decay rate divided by the "noise" or the uncertainty.
% - Because dP/dE is so much smaller than T0 and P, T0 and P need to have pretty small error bars to be able to say we see an actual dP/dE signal
% - We can use equation 35 to estimate expected delta BIC for future observations
% - As we can see in figure 6, delta BIC initially decreases, showing preference for linear model, before increasing. This is because curvature needs to build up over time to see the quadratic term. Depending on the number and frequency of observations, timing uncertainties, and decay rate, we can calculate the crossover epoch, which is the point at which a quadratic model is favored over a linear model. For example, we can choose a desired delta BIC and assign system properties (dP/dE, timing uncertainties) and calculate the minimum number of transits observed before crossover epoch is achieved. So, we can apply this equation to a number of different observing campaign goals and get some info back
% - Overall, can use the analytical delta BIC equation to plan different observing campaigns and see best way to collect data in not only frequency and total number, but also with uncertainties based on ground and space-based scopes (CAN TIE THIS INTO UNCERTAINTIES MAYBE?)
% - Section 3.2 shows real analysis of analytical delta BIC application for 4 different planetary systems. Shows that these systems will not usually reach WASP-12 b levels of delta BIC decay detection until 2030 (TrES-1 b), 2025 (TrES-2 b), 2024 (HAT-P-19 b, although this one is super weird, maybe because of uncertainties??)
% - Caveats: 
%     - used a linear regression approach for the uncertainties (the whole S thing from numerical recipes) and thus analytical delta BIC equation, but real data could have asymmetric uncertainties (QUESTION: is this what Brian is usually talking about when he mentions gaussian uncertainties??)
%     - delta BIC usually only works when sample size is much larger than the number of model parameters (however this is usually fulfilled because to search for decay you need a lot of observations anyways)
%     - doesn't look into precession and/or other forms of changing orbit


\subsubsection{Application to UHJ Systems}
Evaluate the effectiveness of these metrics when applied to UHJs, considering their unique characteristics.

\begin{figure}[htbp]
    \centering
    \includegraphics[width=0.5\linewidth]{figs/yee_fig4.png}
    \caption{Caption}
    \label{fig:enter-label}
\end{figure}

\clearpage


\subsection{Proposed Future Research}

\subsubsection{Advanced Modeling/Theoretical Model Limitations}
Discuss the limitations of current tidal dissipation models in explaining the varying rates of orbital decay observed.

Encourage the development of more sophisticated models that incorporate a wider range of stellar and planetary properties to predict tidal dissipation and orbital decay more accurately.

- Interpolable grid of values of Q* based on stellar age and mass, so we can redo analytical approximations. For example, figure 1 of Jackson. 

- If we are using theoretical models for things like the tidal dissipation rate, such as the models and interpolable grid in weinberg et all 2023, then we need to also focus on constraining stellar properties, as these are not fully constrained for many systems and therefore make using an interpolable grid based on stellar parameters obsolete (we can see how the theoretical values range from adams 2024 paper)

- We also need to focus more on stellar properties and structure for the models themselves. As more research is done on different modes and how different waves such as inertial waves behave inside stars and effect tidal dissipation rates, it will be important to understand the exact structure of these stars. This can also be true for the internal structure of planets, because the tidal dissipation rate due to inertial waves also exists in planets, and if UHJs really do have a large solid core, then this would mean a lot for the behavior of inertial waves. 

- Idk what to write about model limitations honestly. I know that they are there, will need to look more into this, maybe re-review the conclusions in Ogilvie or Weinberg papers

- Creating metrics for different types of orbital motion (apsidal precession, line-of-sight acceleration, look into all the models they had on the paper from the people in the Susie meeting)

- Maybe talk about how something else can be used instead of BIC for metrics?

- Maybe talk about the n-body simulations and that while we need the asteroseismology models, we also need these orbital models as well (although do things like VPLanet also introduce the same models of modes and waves generated in stars to simulate the n-body simulations or is it all just simplified grav interactions?)

\subsubsection{High-Precision Transit Timing/Comprehensive Surveys}
- Highlight the need for more extended and precise observational campaigns to detect subtle orbital decay signals.

- Advocate for the development of more precise transit timing methods to detect minute changes in orbital periods.

- Propose conducting comprehensive surveys of UHJs with consistent observation protocols to gather uniform data on orbital periods.

- Propose one large database like what Elisabeth proposes in her paper.

- Talk about future space telescope missions that may be useful/will collect a lot more transits (maybe the very large telescope or vera rubin, which will collect SO MUCH data and data pipelines need to be built for that to try to get as many light curves as possible!)

\clearpage 


\subsection{Conclusion}

\subsubsection{Summary of Findings}
Recap the key insights gained from synthesizing the articles on tidal decay in UHJ systems.

\subsubsection{Importance of Continued Research}
Emphasize the significance of addressing the identified research gaps to enhance our understanding of tidal interactions and the evolution of UHJ systems.

\clearpage



% \textbf{ADAMS PAPER}

% \begin{figure}[htbp]
%     \centering
%     \includegraphics[width=\linewidth]{figs/adams_fig3.png}
%     \caption{Caption}
%     \label{fig:enter-label}
% \end{figure}

% \clearpage


% \textbf{WINN PAPER}

% \begin{figure}
%     \centering
%     \includegraphics[width=0.5\linewidth]{figs/winn_fig9.png}
%     \caption{Caption}
%     \label{fig:enter-label}
% \end{figure}

% \begin{figure}
%     \centering
%     \includegraphics[width=0.5\linewidth]{figs/winn_fig14.png}
%     \caption{Caption}
%     \label{fig:enter-label}
% \end{figure}

% \clearpage

% % TODO
% - Look more into line-of-sight acceleration
% - Should I fully understand the observatory information (example: IoIo info pg 3 of elisabeths' ppr)
% - What does it mean by observations were mostly taken in R?
% - Limb-darkening coefficients? (brought up in Adams and Jackson)
% - Composite times?
% - Rescaled delta BIC from Adams? Obviously this is important and something I will definitely need to touch on when discussing associated  mid-time errors in timing observations/processing sections 
% - What could a period increase mean?
% - Don't fully understand the 2 tests ran on pg 25 of adams
% - Don't fully understand the Q*' thing at the end (pg 25/26)
% - Stellar dissipation rate may be related to stellar luminosity?

\clearpage

\section{References}
\printbibliography[heading=none]
\end{document}

% Carter et. al NOTES (already used/don't need)
% - light curves can be used to find planet parameters, this uses an analytical approach to estimate these parameters without light curves?
% - transit light curves can tell us about planetary and stellar radii, orbital inclination, and mean density of the star. Can also detect other planets in system by observing changes in orbital params or changes in collections of mid-times
% - Even when it is possible to have numerical values, it can help to have analytical approximations as well as uncertainties and covariances
% - Analytical apprx are good for:
%     - Planning observations. For example: which systems will have most precise measurement of orbital inclination, how many transits do I need to observe before stat error in planetary radius is smaller than systematic err
%     - Speeding up convergence of optimization algos
%     - useful in order-of-mag estimates of observability of subtle transit effects such as: TTVs, precession-induced changes, asymmetry in ingress and egress due to non-zero eccentricity